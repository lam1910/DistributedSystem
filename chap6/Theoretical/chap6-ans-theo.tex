\documentclass[11pt,a4paper]{article}

\usepackage{datetime}
\usepackage{graphicx}
\usepackage{enumitem}
\usepackage{amsmath}

\title{Chapter 6: Synchronization}
\newdate{date}{05}{06}{2020}
\date{\displaydate{date}}
\author{Nguyen Ngoc Lam - 20162316}

\begin{document}
\pagenumbering{gobble}
\maketitle
\newpage
\pagenumbering{arabic}
\tableofcontents
\newpage

\section{Give two examples to demonstrate the importance and the need of synchronization mechanism between processes in distributed systems}
\begin{itemize}
	\item Distributed System is a collection of computers connected via the high speed communication network.
	\item The hardware and software components communicate, coordinate their actions and share their resources to other nodes by message passing.
	\item There is need of proper allocation of resources to preserve the state of resources and help coordinate between the several processes.
	\item Example: If a distributed system is not synchronized, for example computer A is 10 minutes behind computer B and both computer is on the same timezone (say ICT) with computer B is synchronized with the standard time in timezone. At 7:00 am Hanoi time, computer A try to write something to the shared resources and register the timestamp as 7:00 Hanoi time, but since it is technically 7:10 am, the system will not let the user from A to write anything because the timestamp is invalid. In fact, users from computer A cannot write anything until the administrator synchronizes that machine. Timestamp invalidation can also caused lagging when using real time application or application that shared the resources to a lot of users like online games or chat applications.
	\item Benefit of synchronization through example: With synchronization, when playing online games for example, since most packages have to have a timestamp or a related time series to determine the player model action sequences, with synchronization, there can be less lag and enable better quality for player. In fact, nowadays, most online games required you to synchronized before playing. 
\end{itemize}

\section{Compare Network Time Protocol and Berkeley algorithm}
\begin{tabular}{|c|c|}
\hline 
Berkeley Algorithm & Network Time Protocol \\ 
\hline 
Having 1 Master to be standard & Having 3 classes with \\ & class 1 is the highest accuracy and\\ & class 3 is the lowest \\ 
\hline 
Master calculate the & Class 1 is the standard. Class 2 get time \\ average time and discard outliers & from Class 1 and Class 2 server. \\ & Class 3 get time from any server \\ & when they attempted connecting to one \\ 
\hline 
Send time adjustment to clients  & Send timestamp to be changed into \\ 
\hline 
Round trip message & One way message to dictate time \\ 
\hline 
\end{tabular}

\section{What is the typical characteristic of synchronization algorithm for wireless networks}
Since the primary concern of all wireless applications is energy conservation, the typical characteristics of synchronization algorithm for wireless networks are :
\begin{itemize}
	\item It must carefully regiment its frequency of resynchronization, and avoid flooding. 
	\item In addition, the algorithm cannot typically rely on a power-hungry source of real time such as GPS. 
	\item Another characteristic of wireless networks is unexpected and possibly frequent changes in network topology. Thus, a CSA in a wireless medium must continue to function in the face of node failures and recoveries. 
	\item Lastly, many applications in wireless settings are local in nature. That is, only nearby nodes in the network need to participate in some activity. Thus, a desirable property for a CSA is that it closely synchronizes nodes which are nearby, while possibly allowing faraway nodes to be more loosely synchronized.
\end{itemize}

\section{What is the difference between physical synchronization and logical synchronization}
\begin{tabular}{|c|c|}
\hline 
Physical Synchronization & Logical Synchronization \\ 
\hline 
Synchronize the exact timestamp & Synchronize by order of action \\ 
\hline 
Must change the clock in each systems & No need to change the clock \\  based on a standard &\\
\hline 
Maintain the same notion of time & Keep track of information pertaining \\ & to the order of events \\ 
\hline 
Expensive to maintain & Inexpensive to maintain \\ 
\hline 
Inherently inaccurate & Fairly inherently accurate \\ 
\hline 
\end{tabular}

\section{What are the update steps of counters to implement Lamport’s logical clock}
\begin{itemize}
	\item Once the logical clock function has an event, it looks at the time on that event, and compares it to the time on the clock’s process
	\item It chooses the larger of the options, and increments it arbitrarily
\end{itemize}

\section{Answer questions based on an algorithm for the physical clock synchronization}
\begin{enumerate}[label=\alph*)]
	\item Is the value of $T_P$ calculated by the above formula absolutely accurate?
	\begin{itemize}
		\item Given that there is a constant delay on the medium, the value $T_P$ can be consider absolutely accurate because $RTT$ is double the one way time in that condition and $T_P$ will be equal with the timer on server S at that time.
		\item In practice, the value $T_P$ is not absolutely accurate but it can be consider as accurate enough to use.
	\end{itemize}
	\item Let $\delta$ be the deviation of time value and $min$ the minimum time value it takes to transmit a message one-way. Determine the value $\delta$ using only 2 variables $RTT$ and $min$
	\begin{itemize}
		\item If there is no deviation of time value, $RTT = 2*min$
		\item The deviation of time value is the different between $RTT$ and two time the $min$. So the equation would be:
		\[\delta = RTT - 2*min\]
		\item $\delta$ should be positive but it can be negative when you upgrade to a better system and/or better medium. When that happened, you need to update your $min$ value. 
	\end{itemize}
\end{enumerate}

\section{Answer the following questions in regard of using the Vector Clock concept for enforcing causal communication}
\begin{enumerate}[label=\alph*)]
	\item List  two  conditions  the receiving process use to check whether the message satisfies causality.
	\[\bigg\{
	\begin{matrix}
		V_{Pj}[i] = V_S[i] - 1\\
		V_{Pj}[k] \geq V_S[k] \forall k \in [1, 2, 3, ..., n] - \{i\}
	\end{matrix}
	\]
	\item Vector clock values for 4 points $X1$, $X2$, $X3$, and $X4$
	\[X1 = (0,1,0)\]
	\[X2 = (1,1,0)\]
	\[X3 = (2,1,1)\]
	\[X4 = (2,1,2)\]
	\item Message will be kept at the middleware level:
	Message b
\end{enumerate}

\section{What is a mutual exclusion algorithm in a distributed system}
\begin{itemize}
	\item Mutual exclusion: Concurrent access of processes to a shared resource or data is executed in mutually exclusive manner.
	\item Only one process is allowed to execute the critical section (CS) at any given time.
	\item In a distributed system, shared variables (semaphores) or a local kernel cannot be used to implement mutual exclusion.
	\item Message passing is the sole means for implementing distributed mutual exclusion.
	\item Distributed mutual exclusion algorithms is the family of algorithms that deals with unpredictable message delays and incomplete knowledge of the system state.
\end{itemize}

\section{What is the drawback of the centralized algorithm for the mutual exclusion}
\begin{itemize}
	\item If the coordinator crashes, the entire system may go down with it since it is a single point of failure.
	\item If processes normally block after making a request, they cannot distinguish a dead coordinator from "permission denied" since no message comes back.
	\item Bottleneck when deploy on the large system with a single coordinator (scalability)
\end{itemize}

\section{What is the drawback of the distributed algorithm for the mutual exclusion}
\begin{itemize}
	\item Algorithm is suitable only for small group of processes.
	\item It is highly complex due to the need of identify all processes needed.
\end{itemize}

\section{Propose a solution for the problem of lost token in Token Ring mutual exclusion algorithm}
When a process (say) P1 wants to enter into its critical
section, it sends request to the coordinator. If the
coordinator retains the token, it then sends the token to the
requesting process (P1). After getting the token, the
process will send an acknowledgment to the coordinator,
and enters the critical section. During the execution of the
critical section, P1 will continually send an EXISTS signal
to the coordinator at certain time interval, so that, the
coordinator becomes acquainted that the token is alive and
it has not lost. As a reply to every EXISTS signal, the
coordinator sends back an OK signal to that particular
process (P1), so that the process that is executing in its
critical section (P1), gets to know that the coordinator is
alive also.\\
Now, suppose, the coordinator is not receiving the
EXISTS signal from that process P1. Here two cases may
appear:
\begin{enumerate}
	\item The coordinator assumes that the token has lost. Then the coordinator will regenerate a new token and sends it to that
process (P1) and again it starts executing its critical
section.
	\item The process P1 may crash or fail while executing in it
critical section and consequently, the coordinator does not
receiving any EXISTS signal from P1. Hence, the coordinator will identify it as a crashed process and update the ring configuration table and send the UPDATE signal with update information to other processes to update their own ring configuration tables.
\end{enumerate}
Again it may be the case, that the process P1 (which is currently executing in its critical section), is not receiving the OK signal form the coordinator. So, P1 would assume that the coordinator is somehow crashed. At this moment, the process P1 will become the new coordinator and complete the critical section execution. The new coordinator will send a message [COORDINATOR (PID)] to every other process, that it becomes the new coordinator and send the UPDATE signal with update information, to update the ring configuration tables maintained by all other
processes.\\
The algorithm also overcomes the overhead of token
circulation in the ring. If no processes in the ring want to
enter in its critical section, then there is no meaning of
circulating the token throughout the ring. Rather, in this
approach, the coordinator will keep the token, until any
other process requests it.
\begin{enumerate}
	\item The algorithm guarantees mutual exclusion, because at
any instance of time only one process can hold the token and can enter into the critical section.
	\item Since a process is permitted to enter one critical
section at a time, starvation cannot occur. 
\end{enumerate}

\section{How many messages does the system need to vote the coordinator}
When P3 starts the election after node P4 and P7 was broken, it will take the system a total of 10 messages to elect a new coordinator.
\begin{itemize}
	\item The first round, P3 sent 4 ELECTION messages to P4, P5, P6 and P7. P3 would receive 2 ANSWER messages from P5, P6.
	\item The second round, P5 sent 2 ELECTION messages to P6 and P7 and received 1 ANSWER message from P6.
	\item The final round, P6 sent 1 ELECTION message to P7 and received 0 ANSWER message thus making it the coordinator.
\end{itemize}

\section{Answer questions with a system the has N nodes and each node has a status table of other nodes and the whole system on election algorithm}
\begin{enumerate}[label=\alph*)]
	\item How many messages do we need to vote the coordinator
	\[number\_of\_messages = 2(N-i-1) + (i+1)N\]
	\item When a broken node become working again
	\begin{itemize}
		\item That broken node first broadcast a WORKING message to each of the other nodes for update their tables. The other working nodes will send a acknowledge message back for that previously broken nodes to update its table.
		\item If the said node has higher id than the current coordinator, the coordinator will sent the ELECTION message to said node, otherwise, the system will continue to run
	\end{itemize}
\end{enumerate}

\section{Answer questions regrading a mutual exclusion algorithm}
\begin{enumerate}[label=\alph*)]
	\item How many messages does the system need to successfully let a process use SR. 
	\begin{itemize}
		\item Suppose there are $k$ REQUEST messages on the queue already $0 \leq k$ (no upper limit due to Pi can post multiple REQUEST messages). Pi will wait $k$ RELEASE messages.
		\item Pi sends $N-1$ REQUEST messages and receives $N-1$ REPLY messages.
		\item The total amount of messages would be:
		\[k + 2(N-1)\]
	\end{itemize}
	\item The improvement will cut all the unnecessary REPLY messages since the purpose of the REPLY was to reorder the priority queues of all processes to timestamp order. The new algorithms can cut at most $N - 1$ REPLY messages. In the worst case scenario, when $ts_i$ is the latest, it will perform like the original algorithms.
\end{enumerate}
\end{document}